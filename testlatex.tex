\documentclass{article}

%\documentclass[aps,pra,notitlepage,amsmath,amssymb,letterpaper,12pt]{revtex4-1}
\usepackage{amsmath}
\usepackage{amssymb}
\usepackage{titling}
\usepackage{amsthm}
\usepackage{graphicx}
\usepackage{epstopdf}

\epstopdfDeclareGraphicsRule{.gif}{png}{.png}{convert gif:#1 png:\OutputFile}
\AppendGraphicsExtensions{.gif}

%  Below define helpful commands to set up problem environments easily
\newenvironment{problem}[2][Problem]{\begin{trivlist}
\item[\hskip \labelsep {\bfseries #1}\hskip \labelsep {\bfseries #2.}]}{\end{trivlist}}
\newenvironment{solution}{\begin{proof}[Solution]}{\end{proof}}
 
% --------------------------------------------------------------
%                   Document Begins Here
% --------------------------------------------------------------
 
\begin{document}
 
\title{definition of derivative of a function }
\author{Jianhua Li}
\date{\today}

\maketitle

\begin{abstract}
Create a LaTeX file testlatex.tex (using the template in the info repository). In this file write an explanation of what the definition of the derivative $f'(x)$ of a function $f(x)$ means. Include both inline and numbered equations, as well as a proper title, abstract, and section headings. Find a suitable image to illustrate your definition online, and include it as a figure, with proper citation of the source.
\end{abstract}



%\section{Section Title Here} % Specify main sections this way

\section{Definition}
%The derivative of f(x) is defined as the limit:
%\lim_{x\to\infty} f(x) $$= 2x 
%https://apcalcnotebookarjw.wikispaces.com/Definition+of+the+derivative+of+a+function

\begin{align}
  %f(x) = x^{2}
  f'(x_0) = \lim_{x\to x_0} \frac{f(x_0)-f(x)}{x_0-x}\\
  f'(x) = \lim_{h\to\infty} \frac{f(x+h)-f(x)}{h} %it worked
  %f'(x_0) = \lim_{x\to\x_0}\frac{f(x_0)-f(x)}{x_0-x}
\end{align}

%\begin{align}   
	%f'(x_0) &=\lim_{x\to\x_0}\frac{f(x_0)-f(x)}{x_0-x}
   %$$f'(x) &=\lim_{h\to\infty}f(x)\frac{f(x+h)-f(x)}{h}
%\[
%    \binom{n}{k} = \frac{n!}{k!(n-k)!}
%\]
%f(x) &= 2x^2-16x+35\\
%f'(x) &= 2x+2\\
%\[ \lim_{x \to +\infty} \frac{3x^2 +7x^3}{x^2 +5x^4} = 3.\] 

%\end{align}
% Use align environments for equations. The \\ is a newline character. The & is the alignment character.
% Using align* or \nonumber on each line removes equation numbers

Example Illustration:


\begin{figure}[h!] % h forces the figure to be placed here, in the text
  \includegraphics{250px-Lim-secant.png}  % if pdflatex is used, jpg, pdf, and png are permitted
  \caption{A secant approaches a tangent when h approach to 0,Source https://en.wikipedia.org/wiki/Derivative}
  \label{fig:tangent}
\end{figure}

This text should be below the figure unless \LaTeX  decides that a different layout works better.
 
% Repeat as needed
 
\end{document}


